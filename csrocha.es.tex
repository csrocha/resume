
\newcommand{\Q}{{\textsf{Q}\hspace*{-1.1ex}%
  \rule{0.15ex}{1.5ex}\hspace*{1.1ex}}}
\newcommand{\Cuat}{º\Q~}
\newcommand{\actual}{$\infty$}

\firstname{Cristian Sebastian}
\familyname{Rocha}

\title{Curriculum Vitae}
\address{Av. Rivadaivia 9858 8°D}{Buenos Aires, Capital Federal (1407)}
\phone[mobile]{(+54-911)~6800~0269}
\phone[fixes]{(+54-11)~4635~8436}
\email{cristian.s.rocha@jpmorgan.com;csrocha@gmail.com}
\homepage{https://github.com/csrocha}

\extrainfo{CUIT 23-25095454-9 - Fecha de Nacimiento: 4 dic 1975}
\photo[70pt][0pt]{csrocha.png} % The first bracket is the picture height, the second is the thickness of the frame around the picture (0pt for no frame)
\quote{"The important thing, once you have enough to eat and a nice house, is what you can do for others, what you can contribute to the enterprise as a whole." - Donald Knuth}

\begin{document}

\makecvtitle

\section{Antecedentes Docentes}\label{docentes:desde}

\cventry{2\Cuat 2013}
        {Jefe de Trabajos Prácticos, Cargo Simple Interino}
	{Departamento de Computación, Facultad de Ciencias Exactas y Naturales, Universidad de Buenos Aires, Argentina}
	{Buenos Aires, Argentina}
	{}
	{\begin{itemize}
	\item Encargado de la práctica y de los talleres de la materia de grado \emph{Algoritmos y Estructura de Datos II.}
	\end{itemize}}
\cventry{2010-1\Cuat 2013}
	{Ayudante de Primera, Cargo Simple Regular}
	{Departamento de Computación, Facultad de Ciencias Exactas y Naturales, Universidad de Buenos Aires, Argentina}
	{Buenos Aires, Argentina}
	{}
	{\begin{itemize}
	\item Ayudante de la práctica y de los talleres de las materias de grado \emph{Organización del Computador I} e \emph{Ingeniería de Software, Taller I}
	\end{itemize}}
\cventry{2009-2010}
        {Jefe de Trabajos Prácticos, Cargo Simple Interino}
	{Departamento de Computación, Facultad de Ciencias Exactas y Naturales, Universidad de Buenos Aires, Argentina}
	{Buenos Aires, Argentina}
	{}
	{\begin{itemize}
	\item Encargado de la práctica y de los talleres de la materia de grado \emph{Organización del Computador I}.
	\end{itemize}}
\cventry{2002-2009}
	{Ayudante de primera, Cargo Exclusivo Interino}
	{Departamento de Computación, Facultad de Ciencias Exactas y Naturales, Universidad de Buenos Aires, Argentina}
	{Buenos Aires, Argentina}
	{}
	{\begin{itemize}
	\item Ayudante de la práctica y de los talleres de la materia de grado \emph{Organización del Computador I}
	\item Ayudante de la práctica y de los talleres de la materia de grado \emph{Introducción a la Biología Computacional}
	\item Investigación en Biología Computacional. Aplicación de meta-algoritmos a problemas de Genómica. Aplicación de Teoría de la Información a problemas de Proteómica.
	\end{itemize}}
\cventry{2003-2005}
	{Jefe de Trabajos Prácticos, Maestría en Biología Molecular Médica, Categorización CONEAU A (Res. 228/03), Cargo Simple Interino}
	{Departamento de Computación, Facultad de Ciencias Exactas y Naturales, Universidad de Buenos Aires, Argentina}
	{Buenos Aires, Argentina}
	{}
	{\begin{itemize}
	\item Resonsable de la práctiva y de los talleres de la materia de post-grado \emph{Introducción a la Biología Computacional}.
	\end{itemize}}
\cventry{1997-2001}
	{Docente de talleres de informática}
	{Colegio del Pacífico.}
	{Buenos Aires, Argentina}
	{}
	{A cargo de los talleres de \begin{itemize}
	\item Olimpíadas de Informática,
	\item Antihacking,
	\item UNIX,
	\item Páginas Web,
	\item Internet,
	\item Video Juegos y
	\item Office 97 para docentes y alumnos.
	\end{itemize}}

\section{Antecedentes Científicos}\label{cientificos:desde}

\cventry{2009-2011}
	{Miembro del Laboratorio de Investigación y Desarrollo en Nuevas Tecnologías}
	{Gerencia de Sistemas e Inovación Técnica, Administración Nacional de la Seguridad Social}
	{Buenos Aires, Argentina}
	{}
	{Director: Phd. Mastriani, Mario.}
\cventry{2006-2009}
	{Miembro del Laboratorio de Bioinformátia Estructural, UBACYT 2008-2010}
	{Facultad de Ciencias Exactas y Naturales, Universidad de Buenos Aires.}
	{Buenos Aires, Argentina}
	{}
	{Director: Phd. Turjanski, Adrián.}
\cventry{2006-2009}
	{Modelado de sistemas de transducción de señales mediados por MAPKs, PICT~2006~00580}
	{Facultad de Ciencias Exactas y Naturales, Universidad de Buenos Aires.}
	{Buenos Aires, Argentina}
	{}
	{Director: Phd. Turjanski, Adrián.}
\cventry{2\Cuat 2007}
        {Docente Invitado al Laboratorio The Roitberg Group, Computational Nano- and Bio- Physical Chemistry}
	{Quantum Theory Project, University of Florida}
	{Gainesville, FL, USA}
	{}
	{Director: Phd. Roitberg Adrián.}
\cventry{2\Cuat 2007}
	{Visitante a varios grupos de invistagación instituto.}
	{National Institutes of Health}
	{Washington DC}
	{}
	{Visita al director de tesis de doctorado.}
\cventry{2004-2012}
	{Participante del Centro de computación científica de alto rendimiento, PME~2003~00084}
	{Facultad de Ciencias Exactas y Naturales, Universidad de Buenos Aires.}
	{Buenos Aires, Argentina}
	{}
	{Director: Phd. Marshall, Guillermo.}
\cventry{2003-2006}
        {Participante del proyecto Ampliación y optimizacion del Centro de Genomica Aplicada (CeGA), PME~2003~00137}
	{Facultad de Ciencias Exactas y Naturales, Universidad de Buenos Aires.}
	{Buenos Aires, Argentina}
	{}
	{Director: Phd. Levin, Mariano Jorge.}
\cventry{2003-2004}
	{Participante del proyecto Functional interaction between antibodies to ribosomal P proteins of Trypanosoma cruzi and cardiac receptors in chronic Chagas heart disease. HHMI 55003682 (Howard Hughes Medical Institute)}
	{Facultad de Ciencias Exactas y Naturales, Universidad de Buenos Aires.}
	{Buenos Aires, Argentina}
	{}
	{Director: Phd. Levin, Mariano Jorge.}
\cventry{2002-\actual}
	{Participante de la Red Iberoamericana de Bioinformática (RIB)}
	{Red Iberoamericana de Bioinformática}
	{Buenos Aires, Argentina}
	{}
	{El objetivo de la RIB es proporcionar Capacitación, Apoyo a la Investigación, Servicios y Cooperación con Otras Redes en el área de Bioinformática. http://rib.cecalc.ula.ve/}

\section{Congresos y Escuelas}

\cventry{2009}	
	{Presentación de poster. XI Workshop de Investigadores en Ciencias de la Computación}
	{Universidad Nacional de San Juan}
	{Ciudad de San Juan, San Juan, Argentina.}
	{}
	{http://www.wicc2009.com.ar/}
\cventry{2009}
	{Participación en el Laptop Session. Workshop Biología Computacional de Proteínas}
	{Universidad Nacional de Quílmes}
	{Ciudad de Bernal, Buenos Aires, Argentina}
	{}
	{http://www.biologiacomputacional.org/}
\cventry{2008}	
	{Conferencista, 6º Jornada sobre la Biblioteca Digital Universitaria}
	{Universidad de la Plata}
	{Ciudad de la Plata, Buenos Aires, Argentina.}
	{}
	{http://jbdu.fahce.unlp.edu.ar/}
\cventry{2007}	
	{Alumno y Comunicador, Escuela de Simulación Molecular y Nano-estructuras}
	{Unidad de Matemática y Física, Facultad de Ciencias Químicas, Universidad Nacional de Córdoba}
	{Ciudad de Córdoba, Córdoba, Argentina.}
	{}
	{}
\cventry{2007}	
	{Alumno y Comunicador, Segunda Escuela Argentina de Matemática y Biología}
	{BIOMAT. Universidad Nacional de Córdoba}
	{Ciudad de Córdoba, Córdoba, Argentina.}
	{}
	{}
\cventry{2005}	
	{Alumno y Comunicador, Primera Escuela Argentina de Matemática y Biología}
	{BIOMAT. Universidad Nacional de Córdoba}
	{Ciudad de Córdoba, Córdoba, Argentina.}
	{}
	{}
\cventry{2005}	
	{Participante. Congreso Argentino de Físicoquímica y Química Inorgánica.}
	{Asociación argentina de investigación fisicoquímica}
	{Termas de Rio Hondo, Santiago del Estero, Argentina}
	{}
	{}
\cventry{2004}	
	{Participante. Curso de Bioinformática Aplicada Ao Estudo de Genoma de Insetos Vetores}
	{Centro Argentino Brasileño de Biotecnología - Fundación Osvaldo Cruz}
	{Rio de Janeiro, Brazil}
	{}
	{}

\section{Trabajos publicados}
	\bibliographyunit
	\begin{bibunit}[plain]
		\nocite{*}
		\putbib[crocha-pub]
	\end{bibunit}

\section{Tesis dirigidas}

\cventry{Mayo del 2008}
	{Identificacion de detalles estructurales en proteínas mediante el método de búsqueda de subgrafos isomorfos frecuentes. Lic. Areum Lee}
	{Departamento de Computación, Facultad de Ciencias Exactas y Naturales, Universidad de Buenos Aires.}
	{Buenos Aires, Argentina}
	{}
	{\emph{Director:} Phd. Turjanski, Adrián; \emph{Codirector:} Lic. Cristian S. Rocha.}
\cventry{Noviembre del 2009}
	{Alineamiento de Movimientos Vibracionales para el Estudio Evolutivo de las Dinámicas de Proteínas. Lic. Lucila Sanjurjo} 
	{Departamento de Computación, Facultad de Ciencias Exactas y Naturales, Universidad de Buenos Aires.}
	{Buenos Aires, Argentina}
	{}
	{\emph{Director:} Phd. Sebastian Fernandez Alberti; \emph{Codirector:} Lic. Cristian S. Rocha.}

\label{cientificos:hasta}

\section{Antecedentes de Extensión}\label{extension:desde}

\cventry{2008}
	{Desarrollo del Sitio web para el Grupo de Bioinformática del departamento de Química Biológica}
	{Facultad de Ciencias Exactas y Naturales. Universidad de Buenos Aires. }
	{}
	{}
	{}
\cventry{2007}
	{Dialogo con Cristian Rocha, Especialista en Bioinformática. Teoría del Orden.}
	{Publicación en Página 12, Federico Kusko}
	{}
	{}
	{http://www.pagina12.com.ar/diario/ciencia/index-2007-05-02.html}
\cventry{2007}
	{Sitio web para el Grupo de Modelado Molecular del departamento de Química Inorgánica}
	{Facultad de Ciencias Exactas y Naturales. Universidad de Buenos Aires.}
	{}
	{}
	{http://gmm.qi.fcen.uba.ar/}
\cventry{2007}
	{Sitio web para el Centro de Secuenciamiento de la Facultad}
	{Facultad de Ciencias Exactas y Naturales. Universidad de Buenos Aires.}
	{}
	{}
	{http://seq.ege.fcen.uba.ar/}
\cventry{2007}
	{Incubación del proyecto Moldeo Interactive}
	{Incubacen. Facultad de Ciencias Exactas y Naturales. Universidad de Buenos Aires.}
	{}
	{}
	{Tratamiento de contenido audiovisual en tiempo real usando tecnología SIMD. \emph{Socios:} Andrés Colubri, Fabricio Costa, Gustavo Orrillo, Cristian Rocha.}
\cventry{2003}
	{Sitio web para de Bioinformática del grupo de Chagas del INGEBI}
	{INGEBI. Facultad de Ciencias Exactas y Naturales. Universidad de Buenos Aires. }
	{}
	{}
	{http://machi.exp.dc.uba.ar/}
\cventry{2005}
	{Semana de la Computación: Ciencia detras del Juego.}
	{Departamento de Computación. Facultad de Ciencias Exactas y Naturales. Universidad de Buenos Aires.}
	{}
	{}
	{}
\cventry{2004}
	{Semana de la Computación: Compartiendo Internet.}
	{Departamento de Computación. Facultad de Ciencias Exactas y Naturales. Universidad de Buenos Aires.}
	{}
	{}
	{}
\cventry{2004}
	{Coordinador de administración de redes}
	{Departamento de Computación. Facultad de Ciencias Exactas y Naturales. Universidad de Buenos Aires.}
	{}
	{}
	{Encargado de organizar y coordinar las tareas de los
 	administradores de la red a partir de los requerimientos de los usuarios y
 	la posibilidad que brinda la red. Iniciador de los proyectos: Directorio
 	para la unificación de cuentas de usuarios del Departamento de computación,
 	Página web basada en el administrador de contenido Plone, Reorganización
	de la red física del departamento.}
\label{extension:hasta}

\section{Antecedentes Profesionales}\label{profesional:desde}

\cventry{jul 2016-\actual}
	{Associated}
	{JPMorgan Chase}
	{Buenos Aires, Argentina}
	{Soporte, Diseño y Desarrollo de Software}
	{\begin{description}
	\item [Actividad Principal] Adquisición de Requerimientos, Diseño e Implementación para aplicaciones bancarias en Python.
	\end{description}}
	
\cventry{2009-jul 2016}
	{Emprendedor Cooperativo}
	{Cooperativa de Trabajo Moldeo Interactive Limitada.}
	{Buenos Aires, Argentina}
	{Secretario del Consejo de Administración}
	{\begin{description}
	\item [Actividad Profesional] Consultor Analista Senior en sistemas de computación.
	\item [Proyectos]
	\begin{description}
		\item [Moldeo:] Miembro de la comunidad y desarrollador principal. Sistema OpenSource de visualizacion e interactividad en tiempo real.
		\item [Localización de OpenERP:] Miembro de la comunidad y desarrollador principal. Sistema OpenSource para la Planificación de recursos empresariales.
		\item [Observatorio de PyMES de la Universidad de Bolognia:] Colaborador externo.
		\item [MiERP:] Desarrollador principal. Sistemas ERP Online para la Planificación de recursos empresariales.
		\item [Wifi Social:] Desarrollador secundario. Sistema de redes wifi y servicios web de municipios para sus ciudadanos.
		\item [EME:] Desarrollador secundario. Sistema para diseño de proyectos artísticos. Proyecto subsidiado por Fontsoft y ejecutado por la EME - desarrollos culturales.
		\item [SAS ABT Builder:] Desarrollador principal de sistema para la generación de
			tablas ABT a partir de bases de datos con topología estrella. Utilizado
			en Telecom, Banco nacional de Bolivia, Banco Supervielle, Banco Itau.
		\item [Consultor Senior SAS:] Desarrollador y consultor de sistemas y productos SAS. Principalmente Marketing Automation y Banking Intelligent Solutions.
		\item [Referente técnico de la Software Factory de Telecom Personal:] desde Diciembre del 2011 hasta Septiembre 2012.
	\end{description}
	\end{description}}
\cventry{2009-2010}
	{Investigador en el área de Biología Computacional}
	{Administración Nacional de la Seguridad Social}
	{Buenos Aires, Argentina}
	{}
	{Responsable en el Convenio entre el ANSES y el Hospital Italiano para desarrollar
	aplicaciones en la identificación de enfermedades Genéticas. Convenio entre la
	Universidad de Quílmes para el proceso de información de estructuras de proteínas.}
\cventry{Feb 2008-Jul 2008}
	{Desarrollador de Productos Semánticos para Plone}
	{Inter-cultura Consultora S.A.}
	{Buenos Aires, Argentina}
	{}
	{Lider del proyecto de Desarrollo del conjunto de productos Platecom
	orientados a web comunitarias, multilingües y semánticas. Desarrollo
	de herramientas de tesauros y ontologías. http://www.platecom.com/}
\cventry{2002-2009}
	{Docente Investigador}
	{Universidad de Buenos Aires, FCEyN, UBA}
	{Buenos Aires, Argentina}
	{}
	{Ayudante de primera con dedicación exclusiva.
	Docente en la materia obligatoria Organización del Computador I y de la
	materia optativa Introducción a la Biología Computacional.
	Investigador en Bioinformática, desarrollo de aplicaciones y scripts para
	procesamientos en lote, participante activo en el intercambio del Departamento
	de Computación con el laboratorio de Investigación del Chagas del
	Dr. Mariano Levin y Dr. Martin Vazquez. Trabajando en el Grupo de Modelado Molecular
	del Dr. Dario Estrin.}
\cventry{2000-2002}
	{Administrador de Red}
	{Universidad de Buenos Aires, FCEyN, UBA.}
	{Buenos Aires, Argentina}
	{}
	{Administración de servidores de Procesamiento, Páginas Web,
	Archivos, Correo electrónico (Irix, Solaris, Tru64, Linux).
	Programación de scripts en Perl, Tcl/Tk, PHP, sh.}
\cventry{2000}
	{Programador freelance}
	{}
	{}
	{}
	{Implementación de acelerador Web, usando librerias ActiveX para Internet Explorer. Inicio y diseño del cliente.}
\cventry{1997-2000}
	{Asistente de Informática}
	{Colegio del Pacífico.}
	{Buenos Aires, Argentina}
	{}
	{Administrador de la red SAMBA (Windows/Linux) del
	laboratorio. Técnico reparación y armado de PCs. Organizador del taller de
	Internet. Apoyo técnico y didáctico a profesores y alumnos. Asistente en el
	proyecto de la red de los colegios municipales de la ciudad de Buenos Aires.}
\cventry{1996-1997}
	{Programador/Diseñador Multimedia}
	{Laffont Ediciones electrónicas}
	{Buenos Aires, Argentina}
	{}
	{Diseño e implementación de Verbos Conjugados en Castellano.}
\cventry{1996}
	{Programador Multimedia Junior}
	{LVD sistemas}
	{Buenos Aires, Argentina}
	{}
	{Programador Visual BASIC 3.0 y C++ para Windows 3.x. Grabación de CD. Edición de Imagen y Vídeo.}
\label{profesional:hasta}

\section{Calificaciones, Títulos y Otros}\label{otros:desde}

\cventry{2004-2012}
	{Alumno de Doctorado de FCEyN, UBA. Renuncia por cuestiones personales.}
	{Universidad de Buenos Aires.}
	{Buenos Aires, Argentina}
	{}
	{\begin{description}
	\item [Tema] Desarrollo e implementación de algoritmos para la predicción de la estructura de complejos proteína-proteína.
	\item [Director hasta 2009] PhD. Adrián Turjanski. Grupo de trabajo en Modelado Molecular, Depto. Química Inorgánica. FCEyN.	
	\item [Director a partir de 2009] Dr. Luis Gomez Deniz. Universidad de la Palmas de Gran Canarias.
	\item [Expediente de Graduados] Nº 216/04. Admisión 2 de octubre 2004 (Resolución 10.36	Exp. nº 00480884).
	\end{description}}
\cventry{2002}
	{Licenciado en Ciencias de la Computación}
	{Universidad de Buenos Aires.}
	{Buenos Aires, Argentina}
	{}
	{\emph{Tesis de licenciatura:} Búsqueda de repeticiones en un
	modelo de secuenciamiento aleatorio. \emph{Directora:} Dr. Irene loiseau.}
\cventry{2001}
	{Trainer in Bioinformatics - International Training Course on Bioinformatics (Computational Biology) Applied to Genomic.}
	{Oswaldo Cruz Foundation - UNDP/World Bank/WHO Special Programme for Research and Training in Tropical Diseases, TDR}
	{Rio de Janeriro, Brasil.}
	{}
	{http://www.who.int/tdr/publications/tdrnews/news68/bioinformatics.htm}
\cventry{1995}
	{Técnico en Computación - Con orientación en sistemas administrativos.}
	{E.M.E.T. No. 8 - Otto Krause.}
	{Buenos Aires, Argentina}
	{}
	{}

%\item[Idiomas] Inglés. 4to año Adulto. Sin Título. Cultural Inglesa de Buenos Aires.
\label{otros:hasta}

\end{document}
