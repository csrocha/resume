
\newcommand{\Q}{{\textsf{Q}\hspace*{-1.1ex}%
  \rule{0.15ex}{1.5ex}\hspace*{1.1ex}}}
\newcommand{\Cuat}{º\Q~}
\newcommand{\actual}{$\infty$}

\firstname{Cristian Sebastian}
\familyname{Rocha}

\title{Curriculum Vitae}
\address{Av. Rivadaivia 9858 8°D}{Buenos Aires, Capital Federal (1407)}
\phone[mobile]{(+54-911)~6800~0269}
\phone[fixes]{(+54-11)~4635~8436}
\email{cristian.rocha@moldeo.coop}
\homepage{http://www.github.com/csrocha/}

\extrainfo{CUIT 23-25095454-9 - Birth Date 4 dic 1975}
%\photo[70pt][0pt]{csrocha.png} % The first bracket is the picture height, the second is the thickness of the frame around the picture (0pt for no frame)
\quote{"The important thing, once you have enough to eat and a nice house, is what you can do for others, what you can contribute to the enterprise as a whole." - Donald Knuth}

\newcommand{\DC}{Computer Department}
\newcommand{\FCEN}{Faculty of Science}
\newcommand{\UBA}{University of Buenos Aires}
\newcommand{\CS}{MSc in Computer Science}

\begin{document}

\makecvtitle

\section{Education}\label{otros:desde}

\cventry{2004-2012}
	{Delayed PHd Student, \FCEN, \UBA.}
	{\UBA.}
	{Buenos Aires, Argentina}
	{}
	{\begin{description}
	\item [Title] Research in algorithms for protein-protein interaction prediction.
	\item [Directors] PhD. Adrián Turjanski, FCEN - UBA. Phd. Luis Gomez Deniz. Universidad de la Palmas de Gran Canarias.
	\end{description}}
\cventry{2002}
	{\CS}
	{\UBA.}
	{Buenos Aires, Argentina}
	{}
	{\emph{MSc Thesis:} Algorithms to identify Genomics Repetitions in Random Sequencing Methods. \emph{Director:} PHd. Irene loiseau.}
\cventry{2001}
	{Trainer in Bioinformatics - International Training Course on Bioinformatics (Computational Biology) Applied to Genomic.}
	{Oswaldo Cruz Foundation - UNDP/World Bank/WHO Special Programme for Research and Training in Tropical Diseases, TDR}
	{Rio de Janeriro, Brasil.}
	{}
	{http://www.who.int/tdr/publications/tdrnews/news68/bioinformatics.htm}
\cventry{1995}
	{Computer Technician.}
	{E.M.E.T. No. 8 - Otto Krause.}
	{Buenos Aires, Argentina}
	{}
	{}

\section{Skills}
\cventry{Language}
        {English}
        {Good in read, write and talk technician english}
        {}
        {}
        {}
\cventry{}
        {Spanish}
        {Native language}
        {}
        {}
        {}
\cventry{Software Engineering}
        {Object Oriented Design, Design by Contract ; Unit Testing, Integration Testing, Functional Testing ; DER, UML}
        {}
        {}
        {}
        {}
\cventry{Project Management}
        {Critical Path, Gantt, V-Model, Scrum}
        {}
        {}
        {}
        {}
\cventry{Programming}
        {Python, Javascript, R, C, C++, CSH, BASH, MathLab, SAS, SQL, Assembler}
        {Senior}
        {}
        {}
        {}
\cventry{}
        {Java, Haskell, Ada, Eiffel, Tcl, Visual Basic}
        {Junior}
        {}
        {}
        {}
\cventry{Databases}
        {Postgresql, MySQL, Sqlite, SQLServer, MongoDB}
        {}
        {}
        {}
        {}
\cventry{Technologies}
        {CISC, SPARK, ARM, GPU, Tesla}
        {}
        {}
        {}
        {}
\cventry{Others Languages and Protocols}
        {HTML, CSS, XSLT, YAML, CSV, IP, IPv6, ICMP, Name, SMTP}
        {}
        {}
        {}
        {}

	\pagebreak[4]

\section{Professional Background}\label{profesional:desde}

\cventry{2009-\actual}
	{Cooperative Entrepreneur}
	{Cooperativa de Trabajo Moldeo Interactive Limitada.}
	{Buenos Aires, Argentina}
	{Secretary del Administrative Council}
	{\begin{description}
	\item [Professional Activity] Senior Consultant Analyst
	\item [Proyectos]
	\begin{description}
		\item [Moldeo:] Main Developer. OpenSource system to process signal and render to visualize information in real time using GPU.
		\item [Argentine Odoo Localization:] Main Developer. OpenSource Administrative Software.
		\item [Argentine PyMES Observatory and Bolognia University:] Consultant to develop Pool and Survey Automatization. Project Leader in geo-statistics information publisher for latinamerica industry.
		\item [MiERP:] SAAS Administrative Software.
		\item [SAS ABT Builder:] Developer of ABT Builder System for SAS education Argentina.
		\item [Senior SAS Consultant:] Consultor in Marketing Automation y Banking Intelligent Solutions. Tecnical Reference in the Software Factory of Telecom Personal.
	\end{description}
	\end{description}}
\cventry{2009-2010}
	{Research in Technology for Multidisciplinary Applications.}
	{Administración Nacional de la Seguridad Social}
	{Buenos Aires, Argentina}
	{}
	{Responsible in the agreement between ANSES and the Italian Hospital of Buenos Aires to use ANSES computer capabilities to research in Genomics.}
\cventry{Feb 2008-Jul 2008}
	{Project Leader of Semantic product for Plone}
	{Inter-cultura Consultora S.A.}
	{Buenos Aires, Argentina}
	{}
	{Set of Platecom modules for Web Communities with Semantic and Multilingual support, using of Thesaurus and Ontologies structures.  http://www.platecom.com/}
\cventry{2000-2002}
	{Network and System Administrator}
	{\UBA, FCEyN, UBA.}
	{Buenos Aires, Argentina}
	{}
	{Administrator of the Network and diferentes Process Servers. (Irix, Solaris, Tru64, Linux)}
\cventry{2000}
	{Freelance Developer}
	{}
	{}
	{}
	{Developer of a Web Accelerator in ActiveX for Internet Explorer.}
\cventry{1997-2000}
	{Informatics Assistant}
	{Colegio del Pacífico.}
	{Buenos Aires, Argentina}
	{}
	{Network Administrator. }
\cventry{1996-1997}
	{Multimedia Developer}
	{Laffont Ediciones electrónicas}
	{Buenos Aires, Argentina}
	{}
	{Main developer of ``Verbos Conjugados en Castellano''.}
\cventry{1996}
	{Junior Multimedia Developer}
	{LVD sistemas}
	{Buenos Aires, Argentina}
	{}
	{Programer in Visual BASIC 3.0 \& C++ for Windows 3.x.}
\label{profesional:hasta}

	\pagebreak[3]

\section{Teaching Background}\label{docentes:desde}

\cventry{1999-}
        {Teaching Assistant}
	{\DC, \FCEN, \UBA, Argentina}
	{Buenos Aires, Argentina}
	{}
	{\begin{itemize}
	\item I assisted in undergraduate and postgraduate courses for the MSc in Computer Science,
		the MSc in Geology, and the MSc in Molecular Biology for Medicine of University of Buenos Aires.
	\item Some of these course are \emph{Data Structure and Algorithms},
		\emph{Computer Organization and Architecture}, \emph{Data Bases},
		\emph{Software Ingineering}, \emph{Introduction to Computer Science for Geology},
		and \emph{Introduction to Bioinformatics}.
	\end{itemize}}

	\pagebreak[4]

\section{Scientific Background}\label{cientificos:desde}

\cventry{2009-2011}
	{Research and Develop of new Technology Laboratory}
	{Gerencia de Sistemas e Inovación Técnica, Administración Nacional de la Seguridad Social}
	{Buenos Aires, Argentina}
	{}
	{Director: Phd. Mastriani, Mario.}
\cventry{2006-2009}
	{Structural Bioinformatics Laboratory, UBACYT 2008-2010}
	{\FCEN, \UBA.}
	{Buenos Aires, Argentina}
	{}
	{Director: Phd. Turjanski, Adrián.}
\cventry{2\Cuat 2007}
        {Visitor of The Roitberg Group, Computational Nano- and Bio- Physical Chemistry}
	{Quantum Theory Project, University of Florida}
	{Gainesville, FL, USA}
	{}
	{Director: Phd. Roitberg Adrián.}
\cventry{2004-2012}
	{High Performance Computational Center Project, PME~2003~00084}
	{\FCEN, \UBA.}
	{Buenos Aires, Argentina}
	{}
	{Director: Phd. Marshall, Guillermo.}
\cventry{2003-2006}
        {Optimization and Extension of the Applied Genomic Center (CeGA), PME~2003~00137}
	{\FCEN, \UBA.}
	{Buenos Aires, Argentina}
	{}
	{Director: Phd. Levin, Mariano Jorge.}
\cventry{2003-2004}
	{Functional interaction between antibodies to ribosomal P proteins of Trypanosoma cruzi and cardiac receptors in chronic Chagas heart disease. HHMI 55003682 (Howard Hughes Medical Institute)}
	{\FCEN, \UBA.}
	{Buenos Aires, Argentina}
	{}
	{Director: Phd. Levin, Mariano Jorge.}
\cventry{2002-\actual}
	{Iberoamerican Bioinformatics Network (RIB)}
	{Red Iberoamericana de Bioinformática}
	{Buenos Aires, Argentina}
	{}
	{}

	\pagebreak[3]

	\bibliographyunit
	\begin{bibunit}[plain]
		\nocite{*}
		\putbib[csrocha]
	\end{bibunit}

	\pagebreak[3]

\section{MSc Thesis director}

\cventry{Mayo del 2008}
	{Identificacion de detalles estructurales en proteínas mediante el método de búsqueda de subgrafos isomorfos frecuentes. Lic. Areum Lee}
	{\DC, \FCEN, \UBA.}
	{Buenos Aires, Argentina}
	{}
	{\emph{Director:} Phd. Turjanski, Adrián; \emph{Codirector:} Lic. Cristian S. Rocha.}
\cventry{Noviembre del 2009}
	{Alineamiento de Movimientos Vibracionales para el Estudio Evolutivo de las Dinámicas de Proteínas. Lic. Lucila Sanjurjo} 
	{\DC, \FCEN, \UBA.}
	{Buenos Aires, Argentina}
	{}
	{\emph{Director:} Phd. Sebastian Fernandez Alberti; \emph{Codirector:} Lic. Cristian S. Rocha.}

\label{cientificos:hasta}


\end{document}
