\section{Antecedentes de Extensión}

\cventry{2008}
	{Desarrollo del Sitio web para el Grupo de Bioinformática del departamento de Química Biológica}
	{Facultad de Ciencias Exactas y Naturales. Universidad de Buenos Aires. }
	{}
	{}
	{}
\cventry{2007}
	{Dialogo con Cristian Rocha, Especialista en Bioinformática. Teoría del Orden.}
	{Publicación en Página 12, Federico Kusko}
	{}
	{}
	{http://www.pagina12.com.ar/diario/ciencia/index-2007-05-02.html}
\cventry{2007}
	{Sitio web para el Grupo de Modelado Molecular del departamento de Química Inorgánica}
	{Facultad de Ciencias Exactas y Naturales. Universidad de Buenos Aires.}
	{}
	{}
	{http://gmm.qi.fcen.uba.ar/}
\cventry{2007}
	{Sitio web para el Centro de Secuenciamiento de la Facultad}
	{Facultad de Ciencias Exactas y Naturales. Universidad de Buenos Aires.}
	{}
	{}
	{http://seq.ege.fcen.uba.ar/}
\cventry{2007}
	{Incubación del proyecto Moldeo Interactive}
	{Incubacen. Facultad de Ciencias Exactas y Naturales. Universidad de Buenos Aires.}
	{}
	{}
	{Tratamiento de contenido audiovisual en tiempo real usando tecnología SIMD. \emph{Socios:} Andrés Colubri, Fabricio Costa, Gustavo Orrillo, Cristian Rocha.}
\cventry{2003}
	{Sitio web para de Bioinformática del grupo de Chagas del INGEBI}
	{INGEBI. Facultad de Ciencias Exactas y Naturales. Universidad de Buenos Aires. }
	{}
	{}
	{http://machi.exp.dc.uba.ar/}
\cventry{2005}
	{Semana de la Computación: Ciencia detras del Juego.}
	{Departamento de Computación. Facultad de Ciencias Exactas y Naturales. Universidad de Buenos Aires.}
	{}
	{}
	{}
\cventry{2004}
	{Semana de la Computación: Compartiendo Internet.}
	{Departamento de Computación. Facultad de Ciencias Exactas y Naturales. Universidad de Buenos Aires.}
	{}
	{}
	{}
\cventry{2004}
	{Coordinador de administración de redes}
	{Departamento de Computación. Facultad de Ciencias Exactas y Naturales. Universidad de Buenos Aires.}
	{}
	{}
	{Encargado de organizar y coordinar las tareas de los
 	administradores de la red a partir de los requerimientos de los usuarios y
 	la posibilidad que brinda la red. Iniciador de los proyectos: Directorio
 	para la unificación de cuentas de usuarios del Departamento de computación,
 	Página web basada en el administrador de contenido Plone, Reorganización
	de la red física del departamento.}


