\section{Antecedentes Profesionales}

\cventry{jul 2022-\actual}
	{Jefe de Sistema}
	{Fundación Observatorio PyME}
	{Buenos Aires, Argentina}
	{}
	{Como CTO en la Fundación Observatorio PyME, se me han encomendado responsabilidades clave, que incluyen:
	\begin{description}
		\item[Infraestructura de Curación de Datos:] Supervisar el establecimiento y gestión de la infraestructura de curación de datos, garantizando un procesamiento preciso y confiable de los datos para generar indicadores económicos, estructurales y sectoriales relevantes para las PYMEs.
		\item[Desarrollo de Productos:] Liderar el desarrollo de productos innovadores que aprovechen los indicadores generados, como la creación de aplicaciones amigables para el usuario y plataformas en línea para brindar a las partes interesadas acceso a datos relevantes y accionables.
		\item[Estrategia Tecnológica:] Definir y encabezar la estrategia tecnológica de la organización, explorar y adoptar tecnologías de vanguardia, implementar sistemas de software e identificar oportunidades para mejorar la eficiencia y productividad.
		\item[Desarrollo y Mantenimiento:] Supervisar el desarrollo y mantenimiento de sistemas y herramientas sólidos, incluyendo software y hardware, para garantizar operaciones sin problemas y un rendimiento óptimo.
		\item[Liderazgo del Equipo Tecnológico:] Proporcionar un liderazgo visionario al equipo tecnológico, fomentando un equipo de programadores, ingenieros de software, administradores de bases de datos y profesionales de tecnología altamente capacitados y motivados. Establecer objetivos claros, supervisar el rendimiento y brindar orientación para lograr el éxito.
		\item[Seguridad y Privacidad de Datos:] Garantizar la máxima seguridad y privacidad de los datos sensibles de la empresa mediante la implementación de políticas de seguridad de la información, realizar evaluaciones de riesgos y responder rápidamente a posibles incidentes de seguridad.
		\item[Colaboración con Líderes Empresariales:] Colaborar estrechamente con otros líderes empresariales, como el Director General, el Director de Investigación y el Director Financiero de la Fundación, para alinear las estrategias tecnológicas con los objetivos generales de la organización. Presentar informes sobre el impacto de la tecnología y contribuir a iniciativas empresariales más amplias.
		\item[Interacción con Proveedores de Información:] Colaborar con proveedores externos de información para obtener datos relevantes para la generación de indicadores, evaluar y adquirir información de diversas fuentes y optimizar la adquisición y procesamiento de datos.
		\item[Gestión de Proyectos:] Gestionar eficientemente proyectos relacionados con la tecnología, coordinar iniciativas de desarrollo de productos, asignación de recursos y asegurarse de la finalización exitosa y oportuna de los proyectos.
	\end{description}
	La misión del JT es liderar la estrategia tecnológica de la fundación, supervisar el desarrollo y mantenimiento de sistemas y herramientas, dirigir y motivar al equipo de tecnología, garantizar la seguridad y privacidad de los datos de la empresa y colaborar con otros líderes empresariales para lograr los objetivos generales del negocio. Trabajar para impulsar la innovación y la eficiencia mediante la adopción de nuevas tecnologías, la implementación de procesos de prueba y control de calidad y el establecimiento de metas claras para el equipo de tecnología. Esforzarse por asegurar que la estrategia tecnológica de la empresa esté alineada con los objetivos generales del negocio, y que la tecnología sea un catalizador para el crecimiento y el éxito de la empresa en general.
	}

\cventry{abr 2020-jul 2022}
	{Ingeniero de Infraestructura para Inteligencia Artificial y Procesamiento de Datos}
	{Mercado Libre}
	{Buenos Aires, Argentina}
	{}
	{\begin{description}
  \item[Diseño de Infraestructura:] Colaboración con los equipos de desarrollo de IA y científicos de datos para comprender sus necesidades y diseñar la infraestructura adecuada.
  \item[Implementación Técnica:] Desarrollo de plataforma común a todos los equipos de desarrollo de IA.
  \item[Optimización de Rendimiento:] Colaboración con los equipos para optimizar los recursos que se disponen en la plataforma.
  \item[Gestión y Mantenimiento:] Monitoreo, identificación y reparación de los problemas que surgen en la plataforma y en las implementaciones de los usuarios si lo requiren.
  \item[Seguridad:] Adaptación de la plataforma para simplificar la adaptación de los sistemas que corren en ella a las regulaciones nacionales e internacionales sobre seguridad de datos personales. También aplicar las modificaciones necesarias para que asegurar la resilencia a fallas y agujeros de seguridad informática.
\end{description}}


\cventry{jul 2016-mar 2020}
	{Associated}
	{JPMorgan Chase}
	{Buenos Aires, Argentina}
	{Soporte, Diseño y Desarrollo de Software}
	{\begin{description}
	\item [Actividad Principal] Software Reliability Engineering. Adquisición de Requerimientos, Diseño y Programación para aplicaciones de automatización de infraestructura en Bash, Python, REACT, Kafka, Cassandra.
	\item [Actividades Previas] Adquisición de Requerimientos, Diseño y Programación para aplicaciones bancarias en Python y REACT.
	\end{description}}

\cventry{Jul 2014-Jul 2016}
	{Diseñador e Implementador de la Base de Datos de Conocimientos de la Unidad Especializada en casos de apropiación de niños durante el terrorismo de Estado}
	{Ministerio Público Fiscal - Fundación Sadosky}
	{Buenos aires, Argentina}
	{Adquisición de Requerimientos, Diseño de Arquitectura y Desarrollador Full Stack. Puesta en producción y mantenimiento} 
	{\begin{description}
	\item [Plataforma] Odoo.
	\item [Desarrollos] Módulos para: almacenamiento y seguimiento de casos. Campañas de Investigación. Relaciones entre personas. Generación de estadísticas.
	\item [Soporte] Adaptación de términos y vistas.
	\end{description}}

\cventry{2009-jul 2016}
	{Emprendedor Cooperativo}
	{Cooperativa de Trabajo Moldeo Interactive Limitada.}
	{Buenos Aires, Argentina}
	{Secretario del Consejo de Administración}
	{\begin{description}
	\item [Actividad Profesional] Consultor Senior en el Desarrollo en Soluciones Informaticas.
	\item [Proyectos]
	\begin{description}
		\item [Moldeo:] Miembro de la comunidad y desarrollador principal. Sistema OpenSource de visualización e interactividad en tiempo real.
		\item [Localización de OpenERP:] Miembro de la comunidad y desarrollador principal. Sistema OpenSource para la Planificación de recursos empresariales.
		\item [GeoStat:] Plataforma de presentación de información geoestadística para la pequeña y mediana empresa. Diseñador y Desarrollador principal. R, Python, JavaScript, D3JS.
		\item [Observatorio de PyMES de la Universidad de Bolognia:] Colaborador externo. Investigación y desarrollo de algorimos en geoestadística para la clasificación automática de características industriales.
		\item [MiERP:] Desarrollador principal. Sistemas ERP Online para la Planificación de recursos empresariales.
		\item [Wifi Social:] Desarrollador secundario. Sistema de redes wifi y servicios web de municipios para sus ciudadanos.
		\item [EME:] Desarrollador secundario. Sistema para diseño de proyectos artísticos. Proyecto subsidiado por Fontsoft y ejecutado por la EME - desarrollos culturales.
		\item [SAS ABT Builder:] Desarrollador principal de sistema para la generación de
			tablas ABT a partir de bases de datos con topología estrella. Utilizado
			en Telecom, Banco nacional de Bolivia, Banco Supervielle, Banco Itau.
		\item [Consultor Senior SAS:] Desarrollador y consultor de sistemas y productos SAS. Principalmente Marketing Automation y Banking Intelligent Solutions.
		\item [Referente técnico de la Software Factory de Telecom Personal:] desde Diciembre del 2011 hasta Septiembre 2012.
	\end{description}
	\end{description}}

\cventry{2009-2010}
	{Investigador en el área de Biología Computacional}
	{Administración Nacional de la Seguridad Social}
	{Buenos Aires, Argentina}
	{}
	{Responsable en el Convenio entre el ANSES y el Hospital Italiano para desarrollar
	aplicaciones en la identificación de enfermedades Genéticas. Convenio entre la
	Universidad de Quílmes para el proceso de información de estructuras de proteínas.
        Desarrollo de algoritmos y herramientas para el tratamiendo de información biológica y médica.
        Programación en HTTP, JavaScript, Python, C y C++}

\cventry{Feb 2008-Jul 2008}
	{Desarrollador de Productos Semánticos para Plone}
	{Inter-cultura Consultora S.A.}
	{Buenos Aires, Argentina}
	{}
	{Lider del proyecto de Desarrollo del conjunto de productos Platecom
	orientados a web comunitarias, multilingües y semánticas. Desarrollo
	de herramientas de tesauros y ontologías. http://www.platecom.com/
        Desarrollado en HTTP, JavaScript, Python}

\cventry{2002-2009}
	{Docente Investigador}
	{Universidad de Buenos Aires, FCEyN, UBA}
	{Buenos Aires, Argentina}
	{}
	{Ayudante de primera con dedicación exclusiva.
	Docente en la materia obligatoria Organización del Computador I y de la
	materia optativa Introducción a la Biología Computacional.
	Investigador en Bioinformática, desarrollo de aplicaciones y scripts para
	procesamientos en lote, participante activo en el intercambio del Departamento
	de Computación con el laboratorio de Investigación del Chagas del
	Dr. Mariano Levin y Dr. Martin Vazquez. Trabajando en el Grupo de Modelado Molecular
	del Dr. Dario Estrin.
        Desarrollo de algoritmos y herramientas para el manejo de strings - sequencias de ADN.
        Programación en C, C++, Python, CUDA.}

\cventry{2000-2002}
	{Administrador de Red}
	{Universidad de Buenos Aires, FCEyN, UBA.}
	{Buenos Aires, Argentina}
	{}
	{Administración de servidores de Procesamiento, Páginas Web,
	Archivos, Correo electrónico (Irix, Solaris, Tru64, Linux).
	Programación en Python, Perl, Tcl/Tk, PHP, sh.
	Automatización de tareas de infraestructura.
        Desarrollo de plugins en Python para Plone.}

\cventry{2000}
	{Programador freelance}
	{}
	{}
	{}
	{Implementación de acelerador Web, usando librerias ActiveX para Internet Explorer. Inicio y diseño del cliente.}

\cventry{1997-2000}
	{Asistente de Informática}
	{Colegio del Pacífico.}
	{Buenos Aires, Argentina}
	{}
	{Administrador de la red SAMBA (Windows/Linux) del
	laboratorio. Técnico reparación y armado de PCs. Organizador del taller de
	Internet. Apoyo técnico y didáctico a profesores y alumnos. Asistente en el
	proyecto de la red de los colegios municipales de la ciudad de Buenos Aires.}

\cventry{1996-1997}
	{Programador/Diseñador Multimedia}
	{Laffont Ediciones electrónicas}
	{Buenos Aires, Argentina}
	{}
	{Diseño e implementación de Verbos Conjugados en Castellano.}

\cventry{1996}
	{Programador Multimedia Junior}
	{LVD sistemas}
	{Buenos Aires, Argentina}
	{}
	{Programador Visual BASIC 3.0 y C++ para Windows 3.x. Grabación de CD. Edición de Imagen y Vídeo.}


