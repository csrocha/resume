\section{Antecedentes Científicos}

\cventry{2009-2011}
	{Miembro del Laboratorio de Investigación y Desarrollo en Nuevas Tecnologías}
	{Gerencia de Sistemas e Inovación Técnica, Administración Nacional de la Seguridad Social}
	{Buenos Aires, Argentina}
	{}
	{Director: Phd. Mastriani, Mario.}
\cventry{2006-2009}
	{Miembro del Laboratorio de Bioinformátia Estructural, UBACYT 2008-2010}
	{Facultad de Ciencias Exactas y Naturales, Universidad de Buenos Aires.}
	{Buenos Aires, Argentina}
	{}
	{Director: Phd. Turjanski, Adrián.}
\cventry{2006-2009}
	{Modelado de sistemas de transducción de señales mediados por MAPKs, PICT~2006~00580}
	{Facultad de Ciencias Exactas y Naturales, Universidad de Buenos Aires.}
	{Buenos Aires, Argentina}
	{}
	{Director: Phd. Turjanski, Adrián.}
\cventry{2\Cuat 2007}
        {Docente Invitado al Laboratorio The Roitberg Group, Computational Nano- and Bio- Physical Chemistry}
	{Quantum Theory Project, University of Florida}
	{Gainesville, FL, USA}
	{}
	{Director: Phd. Roitberg Adrián.}
\cventry{2\Cuat 2007}
	{Visitante a varios grupos de invistagación instituto.}
	{National Institutes of Health}
	{Washington DC}
	{}
	{Visita al director de tesis de doctorado.}
\cventry{2004-2012}
	{Participante del Centro de computación científica de alto rendimiento, PME~2003~00084}
	{Facultad de Ciencias Exactas y Naturales, Universidad de Buenos Aires.}
	{Buenos Aires, Argentina}
	{}
	{Director: Phd. Marshall, Guillermo.}
\cventry{2003-2006}
        {Participante del proyecto Ampliación y optimizacion del Centro de Genomica Aplicada (CeGA), PME~2003~00137}
	{Facultad de Ciencias Exactas y Naturales, Universidad de Buenos Aires.}
	{Buenos Aires, Argentina}
	{}
	{Director: Phd. Levin, Mariano Jorge.}
\cventry{2003-2004}
	{Participante del proyecto Functional interaction between antibodies to ribosomal P proteins of Trypanosoma cruzi and cardiac receptors in chronic Chagas heart disease. HHMI 55003682 (Howard Hughes Medical Institute)}
	{Facultad de Ciencias Exactas y Naturales, Universidad de Buenos Aires.}
	{Buenos Aires, Argentina}
	{}
	{Director: Phd. Levin, Mariano Jorge.}
\cventry{2002-\actual}
	{Participante de la Red Iberoamericana de Bioinformática (RIB)}
	{Red Iberoamericana de Bioinformática}
	{Buenos Aires, Argentina}
	{}
	{El objetivo de la RIB es proporcionar Capacitación, Apoyo a la Investigación, Servicios y Cooperación con Otras Redes en el área de Bioinformática. http://rib.cecalc.ula.ve/}

